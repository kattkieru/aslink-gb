\documentclass[10pt]{article}
  \usepackage{color}
  \usepackage{graphicx}
  \usepackage{ifpdf}
  \ifpdf
     \DeclareGraphicsRule{*}{mps}{*}{}
  \fi
  \usepackage{listings}
  \usepackage{fancyhdr}

%========================================

\newcommand{\centeredExternalPicture}[2][]{%
  \begin{center}
    \includegraphics[#1]{figures/#2}%
  \end{center}
}

\newcommand{\code}[1]{{\lstset{basicstyle=\normalsize\ttfamily}\lstinline!#1!}}

\newcommand{\dependencyFigure}[3]{%
  % #1: number of figure
  % #2: lowercase module name
  % #3: capitalized module name
  \generalDependencyFigure{#1}{figure:module_dependency_#2}%
                              {Module Dependencies of ``#3'' Module}
}

\newcommand{\definition}[1]{\emph{#1}}

\newenvironment{documentation}{\vspace*{-\medskipamount}%
                               \begin{list}{\topsep 0mm}%
                               \item\relax\small\selectDocuFont}%
                              {\end{list}\smallskip}

\newcommand{\externalPicture}[2][]{%
   \includegraphics[#1]{figures/#2}%
}

\newcommand{\gbar}{\vrule{} }

\newcommand{\generalDependencyFigure}[3]{%
  % #1: number of figure
  % #2: label
  % #3: caption
  \begin{figure}[th]
    \centeredExternalPicture[width=\textwidth]{linkerFigures-#1.mps}
    \caption{#3}
    \label{#2}
  \end{figure}
}

\newenvironment{grammar}{\begin{tabular}{rcp{8cm}}}
                        {\end{tabular}}

\newcommand{\identifier}[1]{\texttt{#1}}

\newenvironment{moduleHeader}{\bgroup \setlength{\parindent}{0pt}
                              \setlength{\parskip}{5pt}\selectDocuFont}%
                             {\egroup~\\}

\newcommand{\moduleRef}[1]{\ref{module:#1}}
\newcommand{\moduleStart}[1]{\subsubsection{Module ``#1''}\label{module:#1}}

\newenvironment{optionList}{\begin{center}\begin{tabular}{p{4cm}p{8cm}}}%
                           {\end{tabular}\end{center}}

\newcommand{\selectCodeFont}{\fontfamily{cmr}\selectfont\small}
\newcommand{\selectDocuFont}{\fontfamily{cmr}\fontshape{sl}
                             \selectfont\small}

\newcommand{\url}[1]{\texttt{\color{blue}#1}}

%========================================
\pagestyle{fancy}
%
% head- and footlines
%
  \fancyhead[LE,RO]{\rightmark}
  \chead{}
  \fancyhead[LO,RE]{Revised Linker Documentation}
  \fancyfoot[LO,RE]{Dr.~Thomas Tensi}
  \cfoot{2009-10-08}
  \fancyfoot[LE,RO]{\thepage}
  \renewcommand{\headrulewidth}{0.4pt}
  \renewcommand{\footrulewidth}{0.4pt}
  \addtolength{\headheight}{\baselineskip}
%

%############################################################
\begin{document}
\lstdefinelanguage[TT]{C}[ANSI]{C}{%
                  morekeywords={define,in,include,inout,out,readonly}%
                  %
                  }
\lstset{language=[TT]C,basicstyle=\selectCodeFont,
        keywordstyle=\bfseries, columns=[c]spaceflexible,
        aboveskip=\medskipamount, belowskip=0pt}

\title{Technical Documentation for the Revised ASxxxx Linker}
\author{Dr.~Thomas Tensi}
\date{2009-10-08}
\maketitle

%=====================
\section{Introduction}
%=====================

\setlength{\parindent}{0pt}
\setlength{\parskip}{5pt}

\subsection{Overview}

This program is a completely renovated version of the linker from the
8-bit-assembler suite ASxxxx from Alan Baldwin.  Parts of this
documentation are based on his suite documentation \cite{Baldwin09}.

The main idea was to modularize the program towards a
plugin-architecture.  All target platforms supported provide several
routines which are called by the linking framework.  Platform specific
code should only go into those modules; platform-independent code
should go into the general framework.

Apart from some syntactical improvements and data structure
enhancements the main contribution of this linker is the support for
bankswitching at link time.  The SDCC compiler has provided
bankswitching support for several platforms at compile time, but this
proves to be too unflexible in practice.  Take a library module for
example: it may go to several banks depending on the bank layout of
the calling program.

The renovated linker uses a configuration file which tells the bank
number for each module.  The linker puts the modules in the
appropriate banks and also takes care that interbank calls are
correctly handled.

%=====================
\section{Usage}
%=====================

The linker is a command line program with the following syntax and
options:

\begin{verbatim}
  aslink [-i] [-s] [-m] [-u] [-x] [-d] [-q] [-b area = value] 
         [-g symbol = value] [-k path] [-l file]
         file ...
\end{verbatim}

%----------------------------------------
\subsection{Platform Independent Options}
%----------------------------------------

The following options are supported:

\begin{optionList}
  file \dots&list of files  to  be linked
\end{optionList}

\paragraph{General Options:}
  \begin{optionList}
    -b area = expression & specifies an area base address via an
                           expression which may contain constants
                           and/or defined symbols from the linked
                           files (not yet supported)\\

    -g symbol = expression & specifies value for the symbol via an
                             expression which may contain constants
                             and/or defined symbols from the linked
                             files
  \end{optionList}


\paragraph{Library File Options:}
  \begin{optionList}
    -k path & specifies a library directory prefix path; this will
              later be combined with a relative library path to find
              complete library path names\\

    -l file & specifies the relative or complete path of a library
              file; when a relative path is given, the absolute path
              is found by combination with some library directory
              prefix path
  \end{optionList}

Note that more than one path and also multiple library files are allowed.


\paragraph{Output Format Options:}
  \begin{optionList}
    -i & linker output is in Intel hex format (to \code{file.ihx})\\
    -s & linker output is in Motorola S19 format (to \code{file.s19})
  \end{optionList}

Note that output format options are cumulative; it is possible to have
the output in more than one format.


\paragraph{Map and List File Options:}
  \begin{optionList}
    -m & generate a map file into \code{file.map} containing a list of
         symbols (grouped by area) with relocated addresses, sizes of
         linked areas, and other linking data\\

    -u & update all available listing files \code{file.lst} into
         \code{file.rst} by replacing relocatable bytes by their
         correct absolute values\\

    -x & use a hexadecimal number radix for the map file\\
    -d & use a decimal number radix for the map file\\
    -q & use an octal number radix for the map file
  \end{optionList}


%---------------------------------------------------
\subsection{Specific Options for Z80/GBZ80 Platform}
%---------------------------------------------------
  \begin{optionList}
    -j              & generate a map file into \code{file.sym} in a form
                      suitable for the debugger within the NO\$GMB Gameboy
                      emulator\\

    -yo number      & set count of ROM banks to \code{number} (default is 2)\\

    -ya number      & set count of RAM banks to \code{number} (default is 0)\\

    -yt number      & set cartridge MBC type to \code{number} (default is
                      no MBC)\\

    -yn name        & set name of program to \code{name} (default is the
                      name of output file truncated to eight characters)\\

    -yp addr = byte & set byte in the output executable file at
                      address \code{addr} to \code{byte}\\

    -z              & produce a Gameboy image as file (with extension
                      \code{.gb})
  \end{optionList}



%=====================
\section{Processing}
%=====================

The linker gathers all command line files and processes them in the
order presented in two passes.

In the first pass all modules, areas and symbols definitions and
references are collected.  For each referenced but undefined symbol
all library files are searched and ---~when they contain such a
symbol~--- are added to the object file list.  This process continues
until no more references can be satisfied.  The library files are
found by doing a combination of prefix paths and relative paths of the
library files.

Finally area and symbol address definitions from the commandline are
added to the internal symbol table.

When banking is used another step has to happen in pass 1: the banking
module of the linker checks for \emph{interbank calls}.  Those
interbank calls go from code in one bank to code in another parallel
bank.  Those calls cannot happen directly, because during the call the
banks have to be switched and also the bank switch upon return from
the callee has to be organized.  For those interbank calls the linker
introduces trampoline code which does the above steps.  Because
banking also needs library modules another search of the libraries has
to be done.

Note that at the end of the first pass it is clear what object files
and libraries are needed for the executable and which symbols are
available.  It is not yet clear where the areas and symbols are
located.

In the second pass all areas are located in the address space.  This
location is done based on the area type which is one of ABS, CON, REL,
OVR or PAG as follows:

\begin{itemize}
  \item Absolute areas (ABS) have a specific address either assembled
        in or given by the \code{-b} option on the linker command
        line.  They are always put to the address given and are not
        relocated.

  \item Relative areas (REL) have a base address of \code{0x0000}
        assembled in.  The real start address is either given by a
        \code{-b} option on the linker command line or is defined for
        the platform.

        All subsequent relative areas are concatenated to proceeding
        relative areas in an order defined by the first linker input
        file.
\end{itemize}

When all areas have been located, their addresses and those of the
contained symbols have been completely defined.

Several checks are done for sanity of the output executable: The
linker e.g. checks whether paged areas (type PAG) are on a 256 byte
boundary and have a length less than or equal to 256 bytes.  Also
referenced but undefined symbols or inconsistent constants'
definitions are reported.

As the main result of linking the output executable is now generated.
The executable format is either one of the standard formats HEX or
S-records or even a platform-specific format (like the Gameboy load
format).  The linker can even produce several output formats in
parallel.

In addition to the output executables, two more kinds of output can be
produced:

\begin{itemize}
  \item A \emph{linking map file} provides detailed information about
        symbol addresses, areas, modules, libraries and errors during
        linking.

  \item The \emph{updated listing files} are generated from original
        assembler listings (.lst-files).  For each of those files a
        companion file is produced which has all addresses and data
        relocated to their final values.
\end{itemize}


%==============================
\section{Banking Caveats}
%==============================

As mentioned before the linker takes care of transforming
interbank-calls appropriately.  Nevertheless there are some topics
which cannot be correctly handled when an architecture does not use
extended addresses for pointers.  This, for example, applies to the
GBZ80, which only has 16-bit pointers:

\begin{itemize}
  \item an indirect call via a function pointer to a function in
        another bank, and

  \item pointer parameters pointing to another bank.
\end{itemize}

Note that the latter problem can be hard to track down: Take as an
example a constant string (which is normally located in the bank of
its compilation unit) and have its pointer passed to some routine as a
parameter.  When that routine is located in another bank or passes
that parameter directly or indirectly to a routine in another bank,
the access to the string will fail.

There are two remedies for that: either tag those constant strings as
nonbanked ---~which is non-standard and uses up space in the nonbanked
area~--- or copy those constants to nonbanked RAM --~which is tedious
and redundant~---.  Unfortunately in all cases the fact that banking
occurs is visible to the programmer.


%==============================
\section{Architecture Overview}
%==============================

Similar to the original structure of Alan Baldwin's program the
revised linker is organized in several modules according to the
building blocks of the object files.

In the revised linker I have tried to abstract as much as possible
from concrete underlying implementation structures (like linked lists)
but provide access to them only via defined module interfaces.  This
makes the access sometimes a bit clumsier but encapsulates design
decisions and also allows to optimize data structures centrally.

\generalDependencyFigure{1}{figure:module_overview_main}
                           {Module Dependencies of ``Main'' Module}

The top level module is the \definition{Main} module which initializes
and finalizes all other modules.  It scans the commandline and gives
all files found to the parser for a two-pass analysis.  After the
first pass undefined symbol references are resolved via the
\definition{Library} module.

The \definition{Parser} module knows about the syntax of object and
library files and calls the appropriate build routines from other
modules in the first pass for symbol allocation and general
construction of the internal network of related objects.

When banking is used, all interbank calls are modified using stub
symbols during the first pass (where resulting trampoline calls and
symbol definitions go into a synthetic object file).

At the end of the first pass the map file is written (if requested)
using the \definition{MapFile} module.

In the second pass the code is relocated by routines of the
\definition{CodeSequence} module.  This module feeds the
\definition{CodeOutput} module which builds some internal
representations for output of code in several formats.  Note that it
is possible to have several formats built in parallel (like
e.g. S-records and Gameboy format).

At the end of the second pass ---~if requested~--- the
\definition{ListingUpdater} module traverses all available assembler
listings and inserts the code data gathered by linking.  This module
is not very clever and heavily relies on the structure of the listing
files currently provided by the SDCC compiler.

Figure~\ref{figure:module_overview_main} shows all modules of the
linker and the usage dependencies of main.  In the course of the
document other usage dependencies will be illustrated where red arrow
lines show dependencies from interface to another interface and black
arrow lines show dependencies from implementation to another
interface.


%==========================
\section{Module Interfaces}
%==========================

This section defines the module interfaces in detail.  It is directly
generated from the module include files leaving out implementation
details exposed there.

All those interfaces follow an object-based approach.  Whenever a
module implements a type, there are routines to make instances of that
type and those which manipulate object of that type (the latter have
an object of the module type as a first parameter).

The naming convention used is as follows:
\begin{itemize}

  \item All names have the module name as a prefix with single
        underscore for externally visible names and two underscores
        for internal names with file scope.

  \item The main type of a module is named \code{Type} (like
        e.g. \code{List\_Type}).

  \item The constructor routines are called \code{make} or
        \code{makeXXX}, destructor routines \code{destroy}.

  \item When a module defines an object type with reference-semantics,
        those references point to private structures with some magic
        number.  An object pointer can then be checked for validity by
        a routine \code{isValid}.

  \item If some internal module data must be kept, a module exports
        routines called \code{initialize} and \code{finalize} for
        initial setup and final cleanup.  When initialization is done,
        also finalization must be executed, but a stateless module may
        have no such routines.

\end{itemize}


\generalDependencyFigure{2}{figure:module_overview_globdefs}
                           {Module Dependencies onto ``Globdefs'' Module}

%-------------------------------------
\subsection{Global Definitions Module}
%-------------------------------------

The module \definition{GlobDefs} provides elementary types and
routines used globally in the program.  E.g., a \code{Boolean} type is
defined here or the code templates for a very simple assertion
checking.  Because it is considered to be a globally known module, its
identifiers are \underbar{not} prefixed with the module name (in
contrast to the convention given above), but they are considered
universal.

Figure~\ref{figure:module_overview_globdefs} shows that all modules
use the GlobDefs module.

\input{globdefs}


%------------------------
\subsection{Base Modules}
%------------------------

\generalDependencyFigure{3}{figure:module_overview_base}
                           {Module Dependencies between Base Modules}

Those modules provide elementary services (like e.g. assertion
checking) or base collection types (like e.g. lists).

\begin{itemize}
  \item The \definition{Error} module offers services for dealing with
        errors which are classified into several kinds of criticality
        and may lead to program abortion in case of fatal errors.
        Error output normally goes to stderr, but may be redirected to
        any open output file.

  \item The \definition{File} module offers very simple file routines
        like opening, closing, reading and writing a file.  For
        convenience several formatting routines are provided for
        numerals, strings and C-strings.  The printf-style of C is
        intentionally not supported.

        For handling embedded information within a file a special
        convention is introduced: when a file name before the
        extension ends with `@' and a decimal number \(n\) this tells
        that the information starts at seek position \(n\).  In the
        linker this will be needed for object file libraries.

  \item The module \definition{Set} is for set operations and it is
        implemented based on bitmaps.  It can only handle sets of few
        integer elements between 0 and 31, but this is okay for our
        purposes.  It has a value-based semantics so sets can be
        copied or passed as parameters like scalar types.

  \item The \definition{String} module encapsulates services for
        strings.  Strings are implemented as pointers to dynamically
        sized memory areas and those strings can be created, copied,
        filled from a C-string, searched, subscripted and so on.  Note
        that strings have a reference-based semantics: directly
        assigning a string to another leads to an alias.  Therefore
        the module has an explicit assignment operation which should
        be used whenever a copy of a string is needed.  The
        \definition{StringList} module only provides construction and
        printout of a list of strings; all other services can be taken
        from the \code{List} module.

  \item The \definition{TypeDescriptor} module is needed by all
        generic collections (like lists and maps).  A \definition{type
        descriptor} is a structure that tells how some element in the
        container is constructed, destroyed, assigned, identified by
        some key and hashed.  For normal scalar types those operations
        are trivial, for reference-based types some function pointers
        have to be stored in the structure.  Whenever such an element
        is used in a collection container, the container knows how to
        act when some container element is created, destroyed and so
        on.  When a routine pointer is \code{NULL}, nothing has to be
        done.  A special default descriptor defined in the module
        consists only of NULL pointers and can be used for scalar
        types (which require no special handling).

  \item The \definition{List} module provides services for generic
        lists.  Elements can be concatenated to those lists, removed
        from them, searched for, etc.

        For iteration over a list a \definition{cursor} can be used:
        it is set on the beginning of a list and gives some current
        element.  It can be advanced and finally returns \code{NULL}
        as a signal that no more elements exist.

        A type descriptor for the embedded elements is given upon
        construction of a list.  Consider e.g. a list of strings: when
        a string element is deleted from the list, the list should
        also free the resources of the string to prevent memory leaks.

  \item The \definition{Map} module encapsulates services for generic
        maps, i.e. partial functions from keys of some type to values
        of some type.  Those maps have at most one value for some key.
        As with lists above a type descriptor is needed for keys.  The
        values in the maps are always considered as non-unique
        references and require no specific actions e.g. when a
        key-value-pair is deleted.

        \definition{IntegerMap} is a variant of a map where the values
        are arbitrary integers.  This implementation variant is
        necessary to allow 0 as a value.

  \item The \definition{Multimap} module provides generic multimaps,
        i.e. partial functions from keys of some type to sets of
        values of some type (you can also interpret them as relations
        between the two types grouped by the first relation partner).
        Similar to maps the values are considered reference-based, the
        key type is specified by a type descriptor upon multimap
        construction.

\end{itemize}

\input{error}
\input{file}
\input{integermap}
\input{list}
\input{map}
\input{multimap}
\input{set}
\input{string}
\input{stringlist}
\input{typedescriptor}


%-----------------------------------
\subsection{Linker Specific Modules}
%-----------------------------------

Those modules provide linker-specific services and rely on the base
modules defined in the previous section.

\begin{itemize}

  \item The module \definition{Area} handles areas and segments.
        Those are groups of code or data with similar properties like
        being overlayed or getting assigned to some specific memory
        location.  Areas are abstract groupw while segments are
        concrete instances of areas within some object module.

        Normally there is some current area where all symbols and code
        encountered in a code module is put.

  \item The module \definition{Banking} handles interbank calls by
        introducing trampoline calls in the nonbanked area.  This
        requires some trickery by synthesizing an object file.  The
        mechanism is completely platform-independent, because all
        specifics are delegated by callbacks to the Target module.

  \item The module \definition{CodeOutput} centralizes the output to
        the code files.  Several such output streams are available and
        they may use arbitrary output formats.  Each stream has to be
        registered upon program startup and uses a platform-dependent
        code output routine.  When some code sequence has to be put
        out, CodeOutput dispatches that request to all open output
        streams.  An example of a platform-specific output routine can
        be found in the Gameboy module.

  \item The module \definition{CodeSequence} encapsulates code
        sequences, i.e. byte sequences to be put to some bank at some
        specific address.  One central routine can apply a list of
        simple relocations to a given code sequence, where a simple
        relocation specifies some position in the code sequence, an
        offset value and some indication on how the value pointed to
        has to be combined with the offset (e.g. added).

  \item The module \definition{Library} encapsulates services for
        object file libraries.  Those are searched for in directories
        and under specific names.  A single routine searches all
        matching files for symbol definitions matching unresolved
        symbol references at the end of the first linking pass,
        resolves those and adds the matching files to the code base
        and the symbols to the symbol table for further processing.

  \item The module \definition{ListingUpdater} updates assembler
        listings associated with all link objects files (except for
        libraries) by inserting relocated code at appropriate places.

  \item The module \definition{MapFile} provides services for
        generation of map files to give an overview about the object
        files read, the allocation of symbols and areas and the
        library files used.

        Several map files can be open at once and each may have a
        different routine for output.  Those target files and map
        output routines must be registered at the MapFile module and
        are then automatically fed.

  \item The module \definition{Module} encapsulates the concept of a
        ``module''.  A module is a group of code and data areas
        belonging together and is the root of all related linker
        objects.  Its associated areas and symbols are accessible via
        several routines.

  \item The module \definition{Parser} is responsible for analysing
        tokenized character streams.  Those can be single or list of
        object files where the parser calls other modules to build up
        internal object structures.

        A reduced scan for a simple list of symbols encountered is
        also possible and the analysis of key-value-assignments by
        equations.

  \item The module \definition{scanner} gathers character streams to
        tokens.  Those character streams are given as a
        read-character-routine and the tokens are ---~among others~---
        identifier, numbers and operators.  As typical for parsing,
        tokens may be pushed back onto input to allow for some
        lookahead during parsing.

  \item The module \definition{stringtable} encapsulates two string
        tables (or string lists) containing the global base address
        definitions and the global symbol definitions as strings.

  \item The module \definition{symbol} administers symbols with name,
        associated segment and address.  The internal table also
        stored whether a symbol has been defined, referenced or both.
        To assist banking a symbol may be split into a real and a
        surrogate symbol and finally a list of referenced but
        undefined symbols may be obtained.

\end{itemize}

\dependencyFigure{4}{area}{Area}
\input{area}
\dependencyFigure{5}{banking}{Banking}
\input{banking}
\dependencyFigure{6}{codeoutput}{CodeOutput}
\input{codeoutput}
\dependencyFigure{7}{codesequence}{CodeSequence}
\input{codesequence}
\dependencyFigure{8}{library}{Library}
\input{library}
\dependencyFigure{9}{listingupdater}{ListingUpdater}
\input{listingupdater}
\dependencyFigure{10}{mapfile}{MapFile}
\input{mapfile}
\dependencyFigure{11}{module}{Module}
\input{module}
\dependencyFigure{12}{noicemapfile}{NoICEMapFile}
\input{noicemapfile}
\dependencyFigure{13}{parser}{Parser}
\input{parser}
\dependencyFigure{14}{scanner}{Scanner}
\input{scanner}
\dependencyFigure{15}{stringtable}{StringTable}
\input{stringtable}
\dependencyFigure{16}{symbol}{Symbol}
\input{symbol}

%----------------------------
\subsection{Platform Modules}
%----------------------------

\dependencyFigure{17}{target}{Target}
\input{target}
\dependencyFigure{18}{gameboy}{Gameboy}
\input{gameboy}


%===============================
\section{Appendix: File Formats}
%===============================

%-----------------------
\subsection{Object File}
%-----------------------

The following EBNF-grammar gives the token structure of an object file
used as input to the linker.

Tokens are either operators, identifiers, numbers, newlines and
comments.  White space (like blanks or tabulators) is necessary to
seperate numbers and identifiers.  Note that a newline is no valid
whitespace but a token on its own.

\begin{grammar}
  address&::=&word .\\
  areaCount&::=&number .\\
  areaIndex&::=&word .\\
  areaMode&::=&word .\\
  areaDefinition&::=&areaLine \{ symbolLine \} \{ codeLine relocLine \} .\\
  areaLine&::=&\code{'A'} areaName \code{'size'} number \code{'flags'}
               number newline .\\
  codeLine&::=&\code{'T'} address \{ number \} newline .\\
  endiannessChar&::=&\code{'H'} \gbar \code{'L'} .\\
  headerLine&::=&\code{'H'} areaCount \code{'areas'}
                 symbolCount \code{'global'} \code{'symbols'} newline .\\
  moduleLine&::=&\code{'M'} moduleName newline .\\
  moduleName&::=&identifier .\\
  \underbar{objectFile}&::=&radixLine headerLine moduleLine
                            [ optionsLine ] \{~areaDefinition \} .\\
  optionsLine&::=&\code{'O'}
                          \{ ( identifier \gbar number \gbar operator ) \}
                          newline .\\
  radixChar&::=&\code{'X'} \gbar \code{'D'} \gbar \code{'Q'} .\\
  radixLine&::=&radixChar endiannessChar newline .\\
  relocationData&::=&number .\\
  relocationIndex&::=&number .\\
  relocationInfo&::=&relocationKind relocationIndex relocationData .\\
  relocationKind&::=&number .\\
  relocLine&::=&\code{'R'} areaMode areaIndex \{ relocationInfo \} newline.\\
  symbolCount&::=&number .\\
  symbolLine&::=&\code{'S'} identifier
                 ( \code{'Def'} \gbar \code{'Ref'} ) number newline .\\
  word&::=&number number .\\
\end{grammar}


%-----------------------
\subsection{Library File}
%-----------------------

The following EBNF-grammar gives the token structure of an library file
used as input to the linker.

Note that the revised linker does \emph{not} support the SDCCLIB
libraries containing directly embedded and indexed object files.  The
only files supported are libraries referencing external object files
by name.

\begin{grammar}
  fileName&::=&$\ll$string without newline character$\gg$ newline .\\
  \underbar{libraryFile}&::=&\{ fileName newline \} .\\
\end{grammar}


%===============================
%\section{References}
%===============================
%\renewcommand{\refname}{}

\begin{thebibliography}{Baldwin09}
  \bibitem[Baldwin09]{Baldwin09}
    {Alan Baldwin.} \textit{ASxxxx Cross Assemblers.}
    Kent State University, Kent, Ohio. (2009).
    \url{http://shop-pdp.kent.edu/ashtml/asxxxx.htm}
\end{thebibliography}

\end{document}